

\section{Related Work}

Several existing research projects have aimed to extract answers from knowledge bases~\cite{yahya2012natural,yao2014information}.
In this work, we leverage existing research to demonstrate the utility of probabilistic knowledge bases.
Previous methods rank answers by their compatibility to the question.
This works additionally computes the joint probability of the underlying facts providing an additional dimension to the answer.
Previous works only use deterministic rules they so they are not able to extract a similar trustworthiness score.

% http://nakashole.com/papers/2012-vlds-urdf.pdf
Nakashole and Mitchell~\cite{nakashole2014languageaware} describe a system
called FactChecker that discovers whether facts are believable by observing the
semantics and considering alternatives. 
In this work, we only uses the knowledge base and have no access to the source
knowledge accuracy of the extractions and context.
Such a constraint also separates this work from knowledge bases such as
Knowledge Vault~\cite{dong2014knowledge} that require source information in
response to a user search.


