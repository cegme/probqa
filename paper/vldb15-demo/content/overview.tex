
\section{System Overview}

This demonstration describes a question answering systems that leverages a probabilistic knowledge base.
To introduce this section, we first give a walk through of how a question is evaluated.
We give a detailed description of how answers and truthfulness is generated.
We then discuss the algorithm used to extract the underlying facts of the questions.



%We begin with the interface, we describe each of the different methods
%the users has to interact with with the backend knoweldge base.
%We then describe the Logic layer that does translation and of user actions
%to the back end actions. It also allows the user to see the current current
%status of the system.
%We also describe the probabilistic knowledge base driving the system.
%We describe its schema and the integrated functions.

\subsection{Walk Through}
\label{sec:probqa-walkthrough}
When a user submits a question \(q\), we leverage the SEMPRE system to translate
the question into a SPARQL $s(q)$, the de facto language for querying RDF data stores.
The  SQL-like formalism of SPARQL queries produces a set of subgraph expressions as triples.
We can then use $s(q)$ and extract the supporting triples \(t_{s(q)}\) that
match the subgraph.
These triples are in the form (subject, predicate, orbject), abbreviated as
\(\langle s, p, o \rangle\), and correspond to the facts that support the answer to the question.
To obtain the \(t_{s(q)}\), we evaluate the SPARQL query and materialize the
intermediate triples.
We use \(t_{s(q)}\) to search the probabilistic knowledge base to obtain the closest matching facts.
Each fact \(f \in \mathcal{F}\) is of the form \(R(A,B)\) and a triple \(t_{s(q)}\) is equivalent to a fact \(f\) when 
\( (s,p,o) = (A,R,B) \).
Given the equivalent facts, we use an in-database inference algorithm to determine the joint probability that each fact is correct.
This probability is a truthfulness score that can be paired with each answer to the question.

%In the next few sub-sections we describe each part of the process. We start
%with the interface where the user can enter natural language questions and a
%description of how SEMPRE processes the question (Section~\ref{sec:probqa-interface}).
%Next, we describe how we perform lookups for candidate facts (Section~\ref{sec:probqa-search}).
%We then define our method to derive the joint probability distribution (Section~\ref{sec:probqa-inference}).


%Our goal is to extract the facts that support the answer to a user's question.
%SPARQL queries describe a sub graph to be matched in the knowedge base.
%We extract the triples sub graps from the SPARQL queries and treat these triples as the facts underlying the answer to the question.
%The algorithm for this is below...


\subsection{Truthfulness}
\label{sec:probqa-interface}

When a user submits a natural language utterance \(q\), we use the SEMPRE 2.0 system to transform the utterance to a logical form~\cite{berant2013freebase,berant2013semantic}.
We then translate the logical form to SPARQL for execution over a knowledge base \(\mathcal{D}\);
let \(s(q)\) be a function that transforms a natural language utterance to the SPARQL query.
We then parse the SPARQL query and extract the intermediate triples \( t_{s(q)} = \{\langle s,p,o\rangle_1, \ldots \}\). 
Like the SPARQL query, these triples only specify a template of the facts that are required to evaluate the utterance.
We evaluate the SPARQL query over \(\mathcal{D}\) to obtain an answer \( \alpha \), we map the query template to define the candidate set of triples \( t^\alpha_{s(q)} \).

For each triples in \( t^\alpha_{s(q)} \) we perform a look up in \(\mathcal{D'}\) which may or may not be equivalent to the knowledge base \(\mathcal{D'}\).
If there is an exact match, the triple is mapped to a score of 1.
If there is no exact match in \(\mathcal{D}\), we estimate the probability of
the triple appearing using a straight-forward application of the chain rule
described below.
The intuition behind this weighting is that if a fact does not exist, we would
like to compute the probability that the facts could exist, using statistical
factual inference.
An equation representing this value is as follows,

\begin{equation}
  \label{eq:probqa-weight}
  \omega(s,p,o) = \begin{cases}
    1 & \mbox{if } exists(\langle s,p,o \rangle) \\ 
    \max( \omega(o,p,s), P(s,p,o)) & \mbox{otherwise,}
  \end{cases}
\end{equation}

where \( P(s,p,o) = P(s|p,o)  P(p|o)  P(o) \).
We then estimate the joint probability of a fact existing given the first-order
inference rules.


%\begin{lstlisting}[language=Python,basicstyle=\small,showstringspaces=false,mathescape=true,frame=single,numbers=left,label=probqa-algo,caption={Algorithm for obtaining the information}]
%def answer (q):
%  t ={`q': q,
%      `s': SEMPRE.toLogicalForm (q) }
%  t[`a'] = SEMPRE.toSPARQL (t[`s'])
%  execute ($\mathcal{D}$, t) # Evaluate query
%  # Do fact search
%  t['$\alpha$'] = findMatches ($\mathcal{D}$, t) 
%  t['$\omega$'] = evaluateTriples ($\mathcal{D}$, t) # Equation~\ref{eq:probqa-weight}
%  t['khop'] = kHop ($\mathcal{D}$, t['$\alpha$'])

%  # TODO Perform a weighted combination of kHop and $\omega$

%  return t
%\end{lstlisting}



    %\KwData {$q$}

%\begin{algorithm}
  %\caption{Question answering algorithm \ceg{TODO: change this to the fact search algo}.}
  %\label{alg:probqa-mainalgo}
  %\begin{algorithmic}[1]
    %\Procedure{GetAnswer}{q}
    %\State$t \gets \{\mathit{q}: q\}$
    %\State$t \gets \{\mathit{s}: \text{toLogicalForm}(q) \}$
    %\State$t \gets \{\mathit{a}: \text{toSPARQL}(t.\mathit{s}) \}$
    %\State$t \gets \{\mathit{res}: \text{execute}(\mathcal{D}, t.\mathit{a}) \}$
    %% Add the actuall code for finding matches and evaluating triples
    %\EndProcedure
  %\end{algorithmic}
%\end{algorithm}





%\subsection{Probabilistic Inference}
%\label{sec:probqa-inference}
%\ceg{Quick description of inference computation.}
%Given that we have a set of matching/partially matching facts, give an
%explanation of how joint inference is evaluated.



\subsection{Fact Search}
\label{sec:probqa-search}

%Describe how facts are searched using the database.
Given a candidate fact \(f\) of the form \(\langle s, p, o \rangle\), we search
the database for the top triples that are similar to \(f\).
Some facts contain blank nodes or expect lists or sets of information.
For example, the utterance ``\texttt{Who are Justin Bieber's siblings?}'' produces 
a SPARQL query that contains the following triples: 
\begin{enumerate}
\item[(1)] \(\langle\)\texttt{justin\_bieber, people.person.sibling\_s, ?x1}\(\rangle\)
\item[(2)] \(\langle\)\texttt{?x1, people.sibling\_relationship.sibling, ?x2}\(\rangle\)
\item[(3)] \(\langle\)\texttt{?x2, type.object.name, ?x2name}\(\rangle\)
\end{enumerate}

The answers to the query are \texttt{Jazmyn\_Bieber} and \texttt{Jaxon\_Bieber}.
The extracted query contains one answer variable, \texttt{?x2}, and one intermediate node, \texttt{?x1}.
For each answer node we perform a depth first
search to evaluate the probability of each set of triples that can construct
the full subgraph.
The maximum probability set of triples is returned as the trustworthiness of the answer.

\eat{%
\begin{verbatim}
StatementPattern
  Var (name=_const-bf4ff051-uri, value=http://rdf.freebase.com/ns/en.justin_bieber, anonymous)
  Var (name=_const-19bbcf65-uri, value=http://rdf.freebase.com/ns/people.person.sibling_s, anonymous)
  Var (name=x1)

StatementPattern
  Var (name=x1)
  Var (name=_const-150646f1-uri, value=http://rdf.freebase.com/ns/people.sibling_relationship.sibling, anonymous)
  Var (name=x2)

StatementPattern
  Var (name=x2)
  Var (name=_const-9f556c3d-uri, value=http://rdf.freebase.com/ns/type.object.name, anonymous)
  Var (name=x2name)
\end{verbatim}
}

\begin{algorithm}
\caption{Algorithm for discovering the truthfulness of the answer.}
\label{alg:probqa-truthfulness}
\begin{algorithmic}[1]
\Procedure{TrustWorthiness}{$t_{s(q)}^\alpha, \mathcal{D}$} %\Comment{Candidate triples and Knowledge Base}

\State $G \leftarrow \mathtt{CreateSubGraph}(t_{s(q)}^\alpha)$ \label{algo:line:createsubgraph}

\State $Q \leftarrow \mathtt{AllPairs}(G, t_{s(q)}^\alpha)$ \Comment{Subgraphs that contain all facts} \label{algo:line:allpairs}

\For {$\text{path } \textbf{in } Q$}  \label{algo:line:jointprob-start}
  \State $S \leftarrow S \cup \mathtt{Prob}(\mathtt{Facts}(\text{path}))$ \Comment{Joint Probability of facts}
\EndFor \label{algo:line:jointprob-end}

\State \textbf{return} $\mathtt{max}(S)$
\EndProcedure
\end{algorithmic}
\end{algorithm}

Algorithm~\ref{alg:probqa-truthfulness} describes the algorithm for obtaining the truthfulness score.
Given the evaluated triples, we first perform a topological sort to generate a subgraph containing all facts (Line~\ref{algo:line:createsubgraph}).
Next, we find all possible facts in the data that satisfy the subgraph (Line~\ref{algo:line:allpairs}).
We can then calculate the maximum joint probability of facts from each of the path (Lines~\ref{algo:line:jointprob-start} to~\ref{algo:line:jointprob-end}).

%Describe how results are ranked.
%Describe how new results are discovered.
%Give algorithm, describe how blank nodes are compressed and other fact cleaning tasks.

%\subsubsection{Graph Exploring}


% See other examples: http://www.vldb.org/2014/program/papers/demo/

