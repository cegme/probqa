
\section{Background}

In this section, we describe the in-database incremental probabilistic inference system.

\subsection{Data Set}
In this subsection we describe the Freebase data set and how we obtain the rules for use with the Khop method.

\subsection{Markov Logic Networks}

Markov Logic Networks (MLNs) are the standard method of modeling uncertainty.
MLNs are mare up of a weighted first-order formulae of the form \(\{F_i, W_i\}\),
where \(F_i\) is a logical expression and \(W_i\) is a weight
specifying how likely it is that the formula is true.

For example, the MLN clauses 
\begin{enumerate}
  \item[] \(0.96\) bornInState(Obama, Hawaii)
  \item[] \(1.40\) \( \forall x \in Person, \forall y \in State, \forall z \in Country : \\
     bornInState(x,y) \cap isStateOf(y,z) \rightarrow bornInCountry(x, z)\)
\end{enumerate}
first, state a face that Obama was born in the state of Hawaii.
The second formulate is an inference rule that states that if a person \(x\) is born in a state \(y\), and a state \(y\) is in a part of a country \(z\),
then that person \(x\) is born in the country \(z\).
These formula do not necessarily apply,
the weights of 0.96 and 1.40 specify the strength of the formula; stronger rules have a lower chance of being violated.
Deterministic rules, or rules that can never be violated are given an infinite weights of $\inf$.


\subsubsection{Grounding}

MLNs are a template generating ground factor graphs.
A factor graph is a set of factors \(\Phi = \{ \phi_1, \ldots, \phi_{|\Phi|} \} \),
where each factor \(\phi_i\) is a function \(\phi_i (\mathbf{X}_i)\) over a
vector of random variables \(\mathbf{X}_i\).
\ceg{add figure of ground factor graph}

\cite{chen2014knowledge}

\subsubsection{Rule Generation}



\subsubsection{Inference}



