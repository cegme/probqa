
\section{Background}

In this section, we give the necessary background from the probabilistic knowledge base backed question answering system.
We first describe the data sets that underly this work.
We then give a brief description of Markov Logic Networks followed by our definition of a probabilistic knowledgebase.

\subsection{Data Sets}
\label{sec:probqa-dataset}

In this demo, we use Freebase, Reverb, and NELL as the underlying knowledge bases.
These knowledge bases store collection of facts as (subject, predicate, object) triples,
representing facts related to real-world entities (people, places, and things).
Each knowledge base differs in scale, schema and construction methods.

\noindent\textbf{Freebase.} Freebase is a web-scale, human-crafted, high precision knowledge base of
well-known topics (entities) and facts~\cite{bollacker2008freebase}.
As of current writing, it contains 47.5 million entities
and 2.9 billion facts, spanning a variety of areas including People, Sports, Music, Internet,
Books, etc, organized into domains. The Freebase KB is publicly accessible as RDF triples.

\noindent\textbf{ReVerb.} ReVerb is an automatic knowledge base construction system
that extracts entities and facts from natural language text~\cite{fader2011identifying}.
It is a universal schema extraction system, extracting both entities and predicates (relations).
The extracted tuples are publicly available, and we use
its ClueWeb extractions, which contains 14.7 million facts, annotated by their confidence
and source URLs.

\noindent\textbf{NELL.} NELL is a never-ending knowledge extraction system~\cite{mitchell2015never}.
It extracts facts and updates its knowledge base every day.
As of current writing, it has extracted 84.6 million facts, of
which 2.2 million are believed to have high confidence.
Like ReVerb, each fact is annotated by its confidence and source URLs.
It also contains information on which extraction
algorithms are used to extract each fact.

In both ReVerb and NELL KBs, recent works have tried to mine \emph{first-order inference rules}.
An inference rule states causal relationships among facts, for example, we have
\[
\text{bornIn}(x, y), \text{locatedIn}(y, z) \rightarrow \text{bornIn}(x, z).
\]
This rule states that if a person \(x\) was born in location \(y\), and location \(y\)
is located within another location \(z\), then person \(x\) is also born in location \(z\).
The Sherlock-Holmes system~\cite{schoenmackers2010learning} mines 30,912 inference rules from ReVerb, and NELL also
has 2 thousand rules with an inference engine as one of its extraction component.


\subsection{Markov Logic Networks}

Markov Logic Networks (MLNs) are the standard method of modeling uncertainty.
MLNs consist of a set of weighted first-order formulae \(\{F_i, W_i\}\),
where \(F_i\) is a first-order formula and \(W_i\) is a weight
specifying how likely the formula is true.

For example, the MLN clauses below state two pieces of information:

\vspace{5pt}
\noindent\begin{tabular}{l l}
\(0.96\) & bornInState(Obama, Hawaii)\\
\(1.40\) & $\forall x\in\text{Person}, \forall y\in\text{State}, \forall z\in\text{Country}$:\\
         & bornState$(x,y) \wedge$ isState$(y,z) \rightarrow$ bornCountry$(x,z)$
\end{tabular}

The first clause states that that Obama was born in the state of Hawaii.
The second formulate is an inference rule that states that if a person \(x\) is born in a state \(y\), and a state \(y\) is in a part of a country \(z\),
then that person \(x\) is born in the country \(z\).
These formula do not necessitate that the formula apply.
The weights of 0.96 and 1.40 specify the strength of the formula; stronger rules have a lower chance of being violated.
Deterministic rules, or rules that can never be violated are given an infinite weights of \(\infty\).


\subsubsection{Grounding and Inference}

MLNs are a template generating ground factor graphs.
A factor graph is a set of factors \(\Phi = \{ \phi_1, \ldots, \phi_{|\Phi|} \} \),
where each factor \(\phi_i\) is a function \(\phi_i (\mathbf{X}_i)\) over a
vector of random variables \(\mathbf{X}_i\).
%\ceg{Maybe add figure of ground factor graph}
The random variables correspond to facts (\(F_i\)) and factors a correspond to rules.

We use the term grounding to refer to the processes of creating the factor graph from an
MLN and a set of clauses.
Each node in the factor graph is a ground atom and has a boolean variable that represents its truth value.
We an perform the grounding step inside the database using a simple series of database queries~\cite{chen2014knowledge}.

For each possible grounding of formula \(F_i\) we create a ground factor
\(\phi_i(\mathbf{X}_i)\) with a value of 1 if the grounding is true, otherwise
\(e^{W_i}\). The marginal probability distribution of a set of grounded atoms \(\mathbf{X}\) is defined as
\begin{equation}
\label{eq:probqa-marginal}
P(\mathbf{X} = x) = \frac{1}{Z} \prod_i \phi_i (\mathbf{X}_i) = \frac{1}{Z} \text{exp} \left( \sum_i W_i n_i(x) \right),
\end{equation}
where \(n_i(x)\) is the number of true groundings of rule \(F_i\) in x, \(W_i\) is its weight, and \(Z\) is the partition function.
This probability gives use the probability of one particular state of a knowledge base.


%\subsubsection{Rule Generation}



%\subsubsection{Inference}


\subsection{Probabilistic Knowledge Bases}

We use a definition of probabilistic knowledge bases derived in previous work~\cite{chen2014knowledge}.
A probabilistic knowledge base is a 5-tuples \(\Gamma = (\mathcal{E}, \mathcal{C}, \mathcal{R}, \Pi, \mathcal{L})\), where
\begin{enumerate}[itemsep=2pt,topsep=3pt,parsep=0pt,partopsep=0pt,
                  leftmargin=0pt,labelindent=0pt,itemindent=12pt]
\item \(\mathcal{E} = \{ e_1, \ldots, e_{|\mathcal{E}|} \} \) is the set of entities.
Each entitie \( e \in \mathcal{E} \) refers to a real-world object.

\item \(\mathcal{C} = \{ c_1, \ldots, c_{|\mathcal{C}|} \} \) is the set of classes (or types).
Each class \( C \in \mathcal{C} \) maybe be a subset of \(\mathcal{E} : C \subseteq \mathcal{E}\), or an unknown class.

\item \(\mathcal{R} = \{ R_1, \ldots, R_{|\mathcal{R}|} \} \) is the set of relations.
Each \(R \in \mathcal{R} \) defines a binary relation on \(C_i, C_j \in \mathcal{C}: R: \subseteq C_i \times C_j\).
We call \(C_i, C_j\) the domain and range of \(R\) and use \(R(C_i,C_j)\) to denote the relation and its domain and range.

\item \(\Pi = \{(r_1, w_1), \ldots, (r_{|\Pi|}, w_{\Pi|})\} \) is a set of weighted facts.
For each \( (r,w) \in \Pi\), \(r\) is a tuple  \((R,x,y)\),
where \(R(C_i,C_j) \in \mathcal{R}, x \in C_i \in C, y \in C_j \in C\), and \((x,y) \in R\);
\(w \in \mathbb{R}\) is a weight indicating how likely it is that \(r\) is true. 

\item \(\mathcal{L} = \{(F_1,W_1),\ldots, (F_{|\mathcal{L}|}, W_{|\mathcal{L}|}) \} \) is a set of weighted rules.
For each \((F, W) \in \mathcal{L} \), \(F\) is a first-order logic clause, and
\(W \in \mathbb{R} \) us a weight indicating how likely the formula \(F\)
holds. 

\end{enumerate}


\subsubsection{Question Answering}

To answer questions, a mapping must be developed between a known
natural language utterance and a subset of all possible facts.
We assume that the knowledge base contains the complete set of facts necessary
to find the answer to the question.
Additionally, in order to quantify the confidence of each fact, it is
important to enumerate the facts that support the final answer.


In this work, we leverage a question answering system named SEMPRE which uses supervised
learning of question answer pairs to create a Lambda Dependency-Based
Compositional Semantics language (\(\lambda\)-DCS)~\cite{berant2013semantic}.
The translation to a logical form, such as a \(\lambda\)-DCS, allows the
semantics to be executed and produce a denotation, or an answer to the utterance.
The \(\lambda\)-DCS can be translated to a SPARQL query for execution over 
Freebase or a similar knowledge base where it can be evaluated.
To achieve this, SEMPRE needs to be provided training examples specific to each source answer.
%In the following sections, we describe how the questions answering system uses
%facts from the probabilistic knowledge base to derive truthfulness to each
%candidate answer.



